\documentclass[a4paper,10pt]{article}
%\documentclass[a4paper,10pt]{scrartcl}

\usepackage[utf8]{inputenc}

\title{TP2 ACT}
\author{Matthieu Caron et Armand Bour}
\date{vendredi 25 septembre 2015}

\pdfinfo{%
  /Title    (TP2 ACT)
  /Author   (Matthieu Caron et Armand Bour)
  /Creator  (Matthieu Caron et Armand Bour)
  %/Producer ()
  %/Subject  ()
  %/Keywords ()
}

\begin{document}
\maketitle

\paragraph{Question 1}
La premiere polyligne ne représente pas une ligne de toit puisqu'elle possède une ligne en diagonale.\newline
La seconde polyligne est bien une ligne de toit.\newline
La troisieme incorrecte.\newline
La quatrieme incorrecte.\newline
\paragraph{Question 2}
La condition pour qu'une polyligne soit une ligne de toit, deux couples consecutifs doivent avoir un x ou un y en commun.
L'idée est d'alterner l'égalité entre les x et les y.
Par exemple, si on a le couple (x,y1)(x,y2) alors le point suivant sera (x1,y2) 
\paragraph{Question 3}
Ça correspond à la ligne de toit de la figure B.
\paragraph{Question 4}
On distingue 3 phases : \newline




\end{document}
