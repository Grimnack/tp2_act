\documentclass[a4paper,10pt]{article}
%\documentclass[a4paper,10pt]{scrartcl}

\usepackage[utf8]{inputenc}

\title{TP2 ACT}
\author{Matthieu Caron et Armand Bour}
\date{vendredi 25 septembre 2015}

\pdfinfo{%
  /Title    (TP2 ACT)
  /Author   (Matthieu Caron et Armand Bour)
  /Creator  (Matthieu Caron et Armand Bour)
  %/Producer ()
  %/Subject  ()
  %/Keywords ()
}

\begin{document}
\maketitle

\paragraph{Question 1}
La premiere polyligne ne représente pas une ligne de toit puisqu'elle possède une ligne en diagonale.\newline
La seconde polyligne est bien une ligne de toit.\newline
La troisieme n'est pas une ligne de toit.\newline
La quatrieme n'est pas une ligne de toit.\newline
\paragraph{Question 2}
La condition pour qu'une polyligne soit une ligne de toit, deux couples consecutifs doivent avoir un x ou un y en commun.
L'idée est d'alterner l'égalité entre les x et les y.
Par exemple, si on a le couple (x,y1)(x,y2) alors le point suivant sera (x1,y2). 
\paragraph{Question 3}
Il correspond à la ligne de toit de la figure B.
\paragraph{Question 4}
On distingue 4 phases : \newline
La première phase consiste à remplir le tableau. Elle a une complexité de $O(x*y)$ avec $x,y$ la taille en abscisse et en ordonnée du tableau respectivement.\newline
La deuxième phase consiste à chercher le début de la ligne de toit. Elle a une complexité de $O(x)$ on commence en $(0,0)$ et 
on parcourt le tableau horizontalement en incrémentant les abscisse jusqu'à tomber sur une case avec la valeur $True$.\newline
La troisième phase consiste à se déplacer dans le tableau vers le haut ou vers la droite. Par défaut, on se déplace vers le haut jusqu'à que la case suivante aie la valeur $False$, 
auquel cas on commence à se déplacer vers la droite, et, de façon similaire, jusqu'à que la case suivante aie la valeur $False$. 
Dans ce cas, soit la case supérieure a également la valeur $False$, auquel cas on passe à la quatrième phase, soit elle a la valeur $True$, auquel cas on reprend la troisième phase.\newline
La quatrième phase consiste à se déplacer dans le tableau vers le bas, jusqu'à que la case suivante aie une valeur de $False$, auquel cas on reprend la troisième phase.\newline
L'algorithme se termine lorsque on a parcouru tout le tableau horizontalement.\newline
Donc la complexité des phases 3 et 4 dépendant de $n$ la longueur de la ligne de toit en $O(n)$.
En conclusion si on compte le remplissage de tableau la complexité se fait en $O(x*y + x + n)$.
\paragraph{Question 5}
Dans ce nouvel algorithme on doit compter la complexité de l'ajout d'immeuble pour ensuite recalculer la ligne de toit.
Pour ajouter un batiment on peut considerer qu'on place simplement les deux points du batiment sur notre graphique. 
On a donc une complexité en $O(2)$


\end{document}
