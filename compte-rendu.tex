\documentclass[a4paper,10pt]{article}
%\documentclass[a4paper,10pt]{scrartcl}

\usepackage[utf8]{inputenc}

\title{TP2 ACT}
\author{Matthieu Caron et Armand Bour}
\date{vendredi 25 septembre 2015}

\pdfinfo{%
  /Title    (TP2 ACT)
  /Author   (Matthieu Caron et Armand Bour)
  /Creator  (Matthieu Caron et Armand Bour)
  %/Producer ()
  %/Subject  ()
  %/Keywords ()
}

\begin{document}
\maketitle

\paragraph{Question 1}
La premiere polyligne ne représente pas une ligne de toit puisqu'elle possède une ligne en diagonale.\newline
La seconde polyligne est bien une ligne de toit.\newline
La troisieme incorrecte.\newline
La quatrieme incorrecte.\newline
\paragraph{Question 2}
La condition pour qu'une polyligne soit une ligne de toit, deux couples consecutifs doivent avoir un x ou un y en commun.
L'idée est d'alterner l'égalité entre les x et les y.
Par exemple, si on a le couple (x,y1)(x,y2) alors le point suivant sera (x1,y2) 
\paragraph{Question 3}
Ça correspond à la ligne de toit de la figure B.
\paragraph{Question 4}
On distingue 4 phases : \newline
La premiere phase conciste à remplir le tableau elle a une complexité de $O(x*y)$ avec $x,y$ la taille en abscisse et ordonnée du tableau.\newline
La deuxieme phase conciste à chercher le début de la ligne de toit qui a une complexité de $O(x)$ on commence en $(0,0)$ et 
on parcours le tableau en incrémentant les abscisse jusqu'à tomber sur une case avec la valeur $True$.
La troisieme phase : on cherche a monter toujours plus haut dans la ligne de toit, si on ne peut pas monter on va a droite, 
enfin si on ne peut pas aller à droite on rentre dans la 4e phase.
La 4e phase on verifie a chaque tour si on peut aller a droite, si on ne peut pas on descend, si on peut on retourne dans la phase 3.
Donc la complexité des phases 3 et 4 dépendant de $n$ la longueur de la ligne de toit en $O(n)$.


\end{document}
